\section{Introduction} \label{sec:intro}
The notion of a game being fair may be made more precise with the concept of symmetry. Broadly speaking we will consider a game fair when the players are indifferent between which position they play, however there are several distinct notions of symmetry that are possible which lead to variations in structure and fairness. For example, the players may or may not care about the arrangement of their opponents. 

This paper surveys the numerous notions of symmetry for finite strategic-form games that are present in the literature, whilst also filling various holes and opening several further directions of research in the area. This is important to our understanding of the theory of symmetric games and fairness, which is fundamental when it comes to the theory of games, artificial intelligence, biology, computer science, economic theory, legal systems, logic, philosophy, political science, along with social choice and voting theory (see for example Arrow's impossibility theorem \cite{arrow1950difficulty, arrow2012social}) to name just a few examples. Note that this paper does not survey the literature on notions of symmetry, though the reader may find it a useful reference if undertaking such an endeavour. 

Symmetry and fairness in the context of games was first explored by von Neumann and Morgenstern \cite{VNM}, outlining what we will later refer to as our label-dependent framework in which player permutations act on strategy profiles, consequently requiring all players have the same strategy labels. Soon after Nash \cite{NashNCG} famously showed that symmetric games have at least one symmetric mixed strategy Nash equilibrium, while more recently Cheng et al. \cite{CRVWSym} showed that fully symmetric $2$-strategy games have at least one pure strategy Nash equilibrium. Notions of symmetry and equivalence also appear in \cite{HarsanyiSelten}.

Notably the term fair has not really been used in the context of non-zero-sum strategic-form games. However the term fair did appear as early as the 1950s in the context of zero-sum games, which are a subclass of strategic-form games, for example see \cite{VNM, gelbaum1959symmetric, tucker1962comb}. There will be a discussion in Subsection \ref{subsec:fairnessdiscussion} defending the author's use of the term fair in the context of symmetric strategic-form games, though note it is an incredibly complicated and intricate topic, with few to no objectively unambiguously correct answers, but may hopefully be helpful with regards to giving people an actual choice in life without threats and other horrible ways of turning victims against each other, rather than rewarding the people who actually cause those problems for society.

Under the theme of anonymity rather than fairness, Brandt et al. \cite{brandt2009symmetries} examined label-dependent notions such as where players are indifferent between who plays which strategy, and where players do not distinguish between their opponents.

A number of people have examined notions of symmetry which may not be captured inside our label-dependent framework, see for example Nash \cite{NashNCG}, Shapley \cite{shapley1960symmetric}, Peleg et al. \cite{peleg1999canonical}, Sudh\"{o}lter et al. \cite{sudholter2000canonical} and Stein \cite{NoahXE}. In order to discuss and analyse such notions we will need to make a detour to examine morphisms between games, the complexity of which has been investigated by Gabarr\'{o} et al. \cite{IsoComplexity}. Inside what will later be referred to as our label-independent framework game automorphisms act on strategy profiles, which also allows players to have distinct strategy labels.

We begin in Section \ref{sec:background} by reviewing numerous mathematical concepts that will play an important role throughout our analysis. In Section \ref{sec:labdepnotions} we survey various label-dependent notions of anonymity and fairness. 

In Section \ref{sec:morphisms} we review game morphisms while showing that game bijections and game isomorphisms form groupoids, which appears to be missing from relevant literature, and introduce matchings as a convenient characterisation of strategy triviality. 

Finally, in Section \ref{sec:labindnotions} we survey various label-independent notions of fairness, discuss how to classify a given game, and outline how to construct and partially order parameterised symmetric games with numerous examples that range over various classes.

